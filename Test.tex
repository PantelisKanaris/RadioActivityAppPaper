\documentclass{article}
\usepackage{graphicx} % Required for inserting images

\title{mdm26-evacuationplanner}
\author{Pantelis Kanaris }
\date{November 2025}

\begin{document}

\maketitle

\tableofcontents

\begin{abstract}

\end{abstract}

\section{Introduction}

\section{Related Work}

\section{System Model \& Problem Formulation}
In this section we present the system model, problem formulation, and baseline approaches.
\subsection{System Model}
The proposed system consists of three principal components: (1) an atmospheric dispersion
simulation module based on the HYSPLIT model, (2) a web-based interface for simulation
configuration and visualization, and (3) an escape-route planning module that operates on the
processed dispersion output.

\textbf{HYSPLIT Simulation Component:}
HYSPLIT is an established atmospheric dispersion model widely used for modelling the
transport and diffusion of hazardous airborne materials. It requires as input the source location,
release height, release duration, and meteorological data corresponding to the simulation period.
Because HYSPLIT only accepts meteorological data in the ARL (Air Resources Laboratory)
format, publicly available GFS (Global Forecast System) data must first be preprocessed using
the official conversion utilities. After preprocessing, the model is executed within a Docker
container hosted on a dedicated server. The model produces a sequence of hourly output files,
each containing concentration values at discrete geographic coordinates.


\textbf{Data Processing Component:}
Upon completion of the simulation, the generated output files are parsed and the concentration
values, coordinates, and timestamps are extracted. These data are then stored on the server in a
structured format for subsequent use by the route-planning module and for visualization.


\textbf{Web Interface Component:}
A web application allows users to configure the simulation parameters (e.g., source location,
release characteristics, meteorological dataset) and initiate model runs. The interface also
provides interactive visualization of the dispersion results on a map, enabling users to explore
the spatial and temporal evolution of the hazardous plume.



\textbf{Escape Route Planner Component:}
The escape-route planning module operates on the processed pollutant concentration data and
the underlying road network stored in a spatial database. It prepares the required exposure and
cost information for subsequent route computation, enabling analysis of evacuation paths under
different dispersion conditions.

\subsection{Problem Formulation}
The evacuation routing problem is defined on a directed road network 
\( G = (V, E) \), where \( V \) denotes the set of intersections and 
\( E \) denotes the set of road segments. Each edge 
\( e \in E \) is associated with a length \( l(e) \) and a 
time-dependent pollutant concentration function \( c(e, t) \), which 
represents the estimated concentration of hazardous material in that
segment at time \( t \).

Given a source node \( s \in V \) and a destination node \( d \in V \),
the objective is to compute a path \( P = (e_1, e_2, \ldots, e_k) \) that \textbf{minimizes cumulative exposure to hazardous material}. In its general form, the optimization problem can be expressed as:

\begin{equation}
    \min \sum_{e \in P} w(e, t),
\end{equation}

where edge cost \( w(e, t) \) is a composite function that can incorporate multiple factors relevant to evacuation routing. In the current system, the cost function includes the following:

\begin{itemize}
    \item \textbf{Exposure cost:} 
    The concentration of pollutant \( c(e, t) \) associated with the edge \( e \)
    at time \( t \).
    \item \textbf{Travel distance:}
    The physical length \( l(e) \) of the road segment.
    \item \textbf{Travel time:}
    A function of distance and expected speed on edge \( e \).
\end{itemize}

The framework is designed to be extensible, allowing additional parameters to be incorporated into the cost function when required. 
Examples of such parameters include:
\begin{itemize}
    \item \textbf{Traffic congestion:}
    Dynamic congestion levels that affect travel time or road usability.
    \item \textbf{Road accessibility constraints:}
    Temporary closures, capacity limits, or restricted zones.
    \item \textbf{Safety or risk modifiers:}
    Penalties for proximity to the release location or hazardous areas.
\end{itemize}

\begin{table}[h]
\centering
\caption{Cost function Used in Evacuation Routing}
\begin{tabular}{|c|p{8cm}|}
\hline
\textbf{Parameter} & \textbf{Description} \\ \hline
Exposure Cost & Pollutant concentration \( c(e, t) \) on edge \( e \) at time \( t \) \\ \hline
Travel Distance & Physical length \( l(e) \) of road segment \( e \) \\ \hline
Travel Time & Estimated travel time on edge \( e \) based on distance and speed \\ \hline
Traffic Congestion & Dynamic congestion levels affecting travel time \\ \hline  
\end{tabular}
\end{table}



Although the mathematical objective remains constant —\emph{ minimizing exposure along a feasible path from \( s \) to \( d \)}—
the specific scenario defined by the user determines the constraints under which the route must be computed. In practice, this system 
supports multiple scenario types, including:
\begin{itemize}
    \item \textbf{Scenario of the user-specified destination:} The user specifies a particular destination \( d \). The route is calculated based on the estimated  travel time from \( s \) to \( d \) and the exposure along the path.
    \item \textbf{Time-constrained scenario:} The user specifies a maximum allowable travel time \( T_{max} \). The route is calculated to minimize exposure while ensuring that the total travel time does not exceed \( T_{max} \).
    \item \textbf{User-prioritized trade-off scenario:} The user assigns a weight between exposure minimization and travel distance/time,modifying the edge cost function.
    \item \textbf{Worst-case scenario (safety-first): the} route is computed under the maximum concentration observed for each edge 
    within the simulation window.
    \item \textbf{Escape route scenario:} The route was computed with no specific destination in mind, but rather to lead the user away from the hazardous area as quickly as possible.
    \item \textbf{Multi-destination scenario:} The user specifies multiple potential destinations, and the system computes routes to each, allowing the user to choose based on exposure and travel time.
    \item \textbf{Predefined safe zones scenario:} The user can select the option of safe zones, and the system computes routes to the nearest safe zone (Government-designated shelters) while minimizing exposure.
\end{itemize}

Regardless of the chosen scenario, the underlying goal of the system is
to compute a path that satisfies the user's constraints while minimizing
the expected exposure to the hazardous material.
\subsection{Baseline Approaches}

\section{System Requirements}

\section{System Arcitecture}

\section{Data Model}

\end{document}
